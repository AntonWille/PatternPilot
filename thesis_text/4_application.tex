% !TeX encoding = UTF-8
\section{Application }

\subsection{Data Collection}

\subsubsection{Choice of Platform}

As the basis for this analysis serve discussions on the QA-platform StackOverflow. StackOverflow is one of the largest
forums for programming, with an explicit focus on technical discussion, discouraging subjective discussions and
“chit-chat”.\cite{stack_overflow_faq}

While on the surface, this might seem like a suboptimal place to investigate culture, as it seems likely to have less
density of discussions on culture than other platforms like mailing lists or hacker-news, it offers a chance to find
those discussions that pop up within the flow of actual problem solving: Practitioners do not join with expectations
to talk about culture, so where these discussions “break out”, the chances are high that it is important.

\subsubsection{Extracting Data}

All publicly accessible data of the StackExchange-Sites is regularly published, and can either be downloaded directly,
or interacted with through the stackAPI. Using the Python Requests-package, a class called StackOverflowAPI was created
with 3 public methods: fetch\_questions, fetch\_answers and fetch\_comments.

Each of these 3 types of posts return a body-field that contains the content of the post, formatted in html, as well
as associated metadata like score, created\_at, and an Author-Object.

Questions have tags associated with them, and of course answers. Answers can have an accepted\_answer-flag, and both
answers and questions can have comments associated with them, as well as a comment count.

To keep the number of requests to the API manageable, batching was used. The process of downloading the data was as follows:
\begin{enumerate}
    \item Download the 100 questions with the highest score, that include the tag respectively ‘Python’ or ‘Ruby’, or ‘Perl’
    \item Download all associated Answers
    \item Create a list of all ID’s, and then download all associated comments as well
    \item Load the data into a local Postgres-Database
    \item Use SQL for data validation
\end{enumerate}

Finally, another Python-Script assembles full documents out of all questions saved in the database in .html-files.
Similar to the Stack Overflow default, answers are sorted by their upvote-count, while comments are displayed
chronologically.\footnote[1]{During encoding we realized that dates are not always completely correct.
According to the official documentation, all dates are in unix epoch time, but some comments posted closely to each other
(and referencing one another) appear in the wrong order.}

For the creation of Codes and Memos, as well as part of the analysis, the qualitative-research tool MaxQDA was used.
The scripts created to enable the data\_export, as well as artifacts created with MaxQDA are available in the
thesis repository. \cite[./scripts]{thesis_repo}

\subsection{Open Coding}

Some initial hints on what to look for were given through the categories derived from our definition of culture,
experience with the PLCs through practice, and the idea that the opinions of the creators might influence the culture
of their communities. Python’s language philosophy is maybe the one most concisely summed up in the collection of
aphorisms called the Zen of Python written by Tim Peters in 2004 (see \ref{appendix:A1}). Indeed during encoding, references
to these aphorisms as a “source of truth” could be observed, especially in the context on how to write idiomatic python code
(as we will see in Chapter \ref{sec:5}).

After a few iterations of doing open coding and attempting to connect and find categories, this early phase of coding
yielded some rough categories, as well as a few “supplementary” codes that could offer insights into related phenomena,
shown in Appendix \ref{appendix:A2}.

Regarding the culture categories, both Truth and Goodness appeared to be implicitly contained in almost every discussion,
which made it very difficult to find specificity in them. On the other hand, statements about efficiency and beauty seemed
to be abundant and specific. For example, multiple people explicitly declared their answers to be one-liners, a very
curious mark of quality for a solution.

The idea of comparing \textit{“One Way vs. Many”}, also proved to be an interesting point of contention, especially within the
Python community, with enough references in both Perl and Ruby posts to be considered a promising direction to invesitage.
An idea that formed here early is that embracing One Way is partially an expression of a wish for authority and order, while approval
of multiple ways is an expression of freedom and individualism, and tracking references to “authority”, or the lack
thereof, could help explain some of the related phenomena.

\textit{“Pragmatism vs. purity”}, the attitude of StackOverflow users in regards to software-quality, and context of development
(eg. personal use vs. professional), and \textit{“Importance of Backwards Compatibility”}, seemed like interesting issues, with their
immediate value not being apparent.

\textit{Idiomatic Code} seemed like a very interesting, yet vague topic as well. Usually, where it was explicitly encountered,
people would use it as a shorthand to divide good code and bad code without detailing what the underlying issues are.

Lastly the so-called “supplementary” category included tracking things like \textit{“Comparisons to other Programming Languages”},
the \textit{“Conversational Tone”} used in answers and comments, as well as usage of packages, libraries and external tools.
\textit{“Humorous remarks”} (jokes), could also potentially provide an interesting window into a community’s culture, but were
mostly tracked as a form of entertainment.

\subsection{Refinement and Axial Coding}

Given the broad topic at hand, despite the technical nature of StackOverflow discussions, there was no shortage of
interesting directions to take, rather the difficulty lay in limiting the kinds of phenomena to observe, in order to
not lose focus. At its largest extent, a total of 109 codes, sorted into various categories were concurrently part
of the investigation. \cite[data/coding\_system]{thesis_repo}

In addition to the already described concepts of the previous sections, a number of new interesting directions
opened up as well:

\subsubsection{Legacy vs. Modern Practices}

The topic of backwards compatibility turned out to be part of a significantly larger topic, namely \textit{“Legacy vs. Modern Practices”}.
While this is in itself an important, contentious and fruitful topic, it also bears importance for some of the other
concepts, as it is often relevant in regards to having multiple different ways of doing things (see Section \ref{sec:5.1}),
and of course is also closely related to ideas on idiomatic code: Styles change over time, and what might have been
considered good practice in the 90s, might not be preferred today.

\subsubsection{Efficiency and Beauty}

Ideas on efficiency and beauty are plentiful in the discussions on Stackoverflow, and many groups of concepts, such as
\textit{“Importance of Readability”}, \textit{“Short Code is better?”}, or \textit{“Focus on Performance”} were found.
These also relate very  closely to questions on idiomatic code, and the kinds of trade-offs made, often relate to each other,
where for example, writing especially fast or short code might clash with readability.

\subsubsection{Authoratative Sources}

The idea of \textit{“Authoritative Sources”} was split up. The usage of external references, be that the official documentation,
tutorials or blog-posts, was tracked in its own category. While this seemed like a good idea at the time, the effort
needed to encode these sources consistently, significantly outweighed their benefit overall.

\textit{“References to abstract rules or principles”}, such as “Never run for "flexibility" until it's impossible to avoid.
One of the unwritten rules of good coding.” (371ruby\_4763121, Pos. 52) were also tracked as an interesting appeal
to ephemeral authority, both an expression of truth and goodness.

\subsubsection{Determining what constitutes One Way}
\label{sec:4.3.4}
In the context of the category of \textit{One Way vs. Many}, a particularly tricky issue was determining what constitutes a
sufficiently different approach. The Code \textit{“Answer gives multiple solutions”} denotes posts that answer the questions
giving multiple distinct approaches, and \textit{“Answer gives option other than Accepted”} denotes posts that differ in
approach from the top answer. Each distinct approach is only counted once, generally by vote count. The heuristic
used to determine if an approach is fundamentally different, is if it uses another data structure, algorithm, control
flow or library to solve the core problem of the related question. However, in some cases the question might be too
vague to have a clear problem resolution. To illustrate the issue, here is one example:

The top answer to the question \textit{“How to determine a Python variable’s type?”} indeed gives two different solutions: One,
you can use the builtin \texttt{type()} function to return the type of a variable, and two, you can check if a variable is of a
given type by using \texttt{isinstance()}.

While these are different approaches, they actually answer different questions, and as such can’t be considered as distinct
solutions for the same problem. However, while this answer offers \texttt{isinstance()} as the preferred function to check a type,
a different answer suggests using assert \texttt{type(variable\_name) == int} which does the same thing as \texttt{isinstance()},
and thus constitutes a different solution to the same problem. Yet another answer suggests using the \texttt{\_\_class\_\_}-property
to check for type, another distinct option, although discouraged by both other commenters and the official documentation.
In fact, sometimes it proved difficult to accept an answer as “just” a different approach, when it seemed like an
objectively worse approach.

\subsection{Selective Coding and Writing}

At this point two core categories - and related categories supplementary codes - were chosen to remain in the selective
coding process, while the rest was “deprecated”. While the topic of “Implicitness vs. Explicitness” showed great promise,
and holds high relevance for practitioners, the complexity exceeded the scope of this thesis and would probably require
additional data sources such as github issues and discussions on mailing lists.

Instead, the focus was set on two questions: What are the attitudes towards One Way vs. Many? And what does it mean to
 write idiomatic code?

After deciding on the most promising categories for investigation, an additional 200 questions per language were extracted,
and different methods were used to find additional relevant data for these categories:
Some documents were chosen based on their title, as the question asked seemed especially relevant to either one of those
categories.  For example, the question \textit{“What is the difference between `\$this`, `@that`, and `\%those` in Perl?”}
was thought to have a high chance to offer insights into attitudes on \textit{One Way vs. Many} and the idea of idioms in Perl.

In addition, keyword searches were performed to find especially relevant sections for specific codes. For example, the
idea of calling code or the process of coding “natural” was encountered a handful of times before, and seemed very
relevant to the question of what constitutes idiomatic code, but the underlying phenomenon, the meaning behind people
saying this remained somewhat unclear. As such, searching for “natural” was helpful in finding relevant data, although
naturally plenty of false positives had to be sorted out first. In the case of using key-word searches, in order to
minimize biases emerging, all instances in the languages were investigated and encoded if appropriate.

Other data was gathered to relate the findings from StackOverflow with the ideas of influential figures in the community,
like interviews, conference speeches and books. This was an iterative process while writing the thesis: Finding references,
relating them, writing, and filling gaps in knowledge and observation through selective encoding and reading the literature.
Indeed, the writing process turned out to be a core exercise in relating the final concepts, and multiple changes to the code system and
the underlying data were necessary.

The core concept of questions surrounding idiomatic code, which we will be discussing in more detail in the next Chapter,
only became apparent at the very end of the writing process, necessitating some changes, and partially explaining
why the data overall is a bit thin.
