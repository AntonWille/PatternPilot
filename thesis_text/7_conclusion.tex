\section{Conclusion}

In this exploratory analysis of cultural differences in the programming language communities of Ruby, Python and Perl,
a number of interesting directions of investigations were opened by taking a GT approach to discussions on the QA-platform
StackOverflow. Due to decisions at the design level, distinct histories and their different contexts of practice,
different pronunciations of culture formed in these communities.

While this thesis does not provide a full GT, and remains inconclusive about causal relationships and definitive cultural
characteristics, it offers some insights into the character of these communities, and can serve as a springboard for
future investigations. A lot of focus was put on disentangling the ideas surrounding the question of if there should be
“ one-- and preferably only one --obvious way to do it”, or many ways.

Ideas on Idiomatic Code emerged as the core category, connecting many of the observed phenomena and a deeper
investigation on what idioms in coding are, how they emerge, and how they are applied could be academically valuable
and of high relevance to industry practitioners.

Overall, while this thesis seems to open more questions that it answers, it can hopefully serve as inspiration for
future researchers, and offer some interesting insight for practitioners in how to communicate and write idiomatic
code in these communities.
