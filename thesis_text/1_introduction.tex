% !TeX encoding = UTF-8
\section{Introduction}

While much attention has been dedicated to dissecting and comparing the structural features of programming languages,
less has been done to analyze the features of the communities forming around them. Similarly, research focused on culture
of the broader tech community has seldomly done so based on the characteristics of the programming languages, but rather
based on other commonalities and differences, like demographics,
engineering context or professional practice.\cite{lenberg_behavioral_2015}.

This thesis aims to find, compare and explain some such differences and their origin, by taking a Grounded Theory Methodology (GT)
approach to discussions on the QA-platform StackOverflow, focusing on three programming languages: Perl, Python and Ruby.
While these 3 languages share many similarities, being first introduced in the same era, primarily as dynamically typed,
high-level, scripting languages, each language also exhibits distinct characteristics that merit closer examination.
Given their large influence and often strong opinions on their respective languages (see Section \ref{sec:2.3}), design decisions
and comments by the creators of their respective languages are of special interest in this investigation,
especially wherever they diverge.

Chapter 2 will introduce core terminology, give some history and background on the languages investigated,
as well as an overview of relevant literature. Chapter 3 gives a theoretical overview of the methodology this thesis is
based on, while Chapter 4 chronicles its application. Chapter 5 then dives into the results, seeking to give some answers
to two intricately related questions: \textit{“Should there be one or multiple ways of doing things in a programming language?”},
usually abbreviated in this thesis as \textit{“One vs. Many”}, and \textit{“What does it mean to write idiomatic code?”}.

Finally, Chapter 6 discusses limitations and gives an outlook on future investigations, while Chapter 7 summarizes the findings.

This exploratory analysis can hopefully serve as inspiration, and give some pointers to practitioners on how to communicate and
frame discussions within either of these communities, and write better, more idiomatic code.
