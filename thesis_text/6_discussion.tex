\section{Discussion}

\subsection{Limitations}

\subsubsection{Data Source}
\label{sec:6.1.1}

A major limitation lies in the main data source for this investigation: StackOverflow is a platform primarily centered
around technical discourse, likely resulting in comparatively lower prevalence of culturally related phenomena compared
to interviews or more casual forums. However wherever these discussions still emerge, despite them explicitly being
discouraged, the chances are good that they are especially relevant.

In addition, compared to data from interviews, books or conference speeches, it is reasonable to assume that less care
is put into many of the answers and comments, although some exceptions could probably be published without the need for
an editor. This does tend to make it harder to fully ascertain the meaning of the writers, and while care was put into
the analysis, some uncertainties remain.

Few things can be inferred about the background and demographics of discussion participants, and novice programmers
share the same voice as professionals. Some research also suggests that StackOverflow is an unfriendly place for women,
compared to other coding-related online platforms. As such, it remains somewhat unclear as to how much the
StackOverflow-Community represents the targeted PLCs at large.

\subsubsection{Data Analysis}

By extracting the questions with the highest vote count over all, the data has a natural bias towards older questions,
with most being asked in 2008/09. These discussions then remain open, and often span until today, where of course older
answers also have a better chance to receive a high vote-count, despite possibly “better” or more “canonical” answers
being posted in the mean-time. Of course communities change over time, and fully accounting for this development
exceeded the scope of this thesis.

A natural bias towards older questions does have one advantage: Python today is significantly more popular according
to most metrics, but this divergence in popularity only started around the time StackOverflow was created. Still,
Python tends to receive significantly more votes, answers and comments on average, with Ruby in the middle and Perl
trailing behind.

In addition, Perl posts tend to be not exclusively tagged as Perl, but often include tags like ‘Shell’ or ‘Awk. Section
5.1 gives some ideas on why this is the case, and why it could be a  more general phenomenon in the Perl community, but
conclusively showing that it is outside the scope of this thesis.

Finally, no active distinction was made between questions on specific frameworks, like Pandas, Django or Ruby On Rails.
The communities around these libraries are often rather distinct subsets of the larger community at hand, and might
warrant distinct analysis. Through the tags associated with questions, these differences could be explored further,
but would probably require significantly more data.

\subsection{Outlook}

The limitations described in the previous section can largely be explained either by the limitations of the data source,
or by broadness of the topic at hand, where some compromises had to be made on the deepness of the analysis.

Future investigations could benefit greatly by extending this kind of investigation with different data sources.
The inclusion of some interviews of the creators benefitted this thesis, and could be expanded in future investigations
to also include sources like conference speeches or language design discussions (eg. Python’s PEP discussions).
While StackOverflow proved fruitful, it mostly includes rather narrow questions, and code-snippets. Looking at both code
repositories and the discussions around them (eg. Github Issues) could be very helpful in explaining how cultural
characteristics play out in a more professional setting.

While exploring concepts surrounding \textit{“Explicitness vs. Implicitness”} was very interesting, and are highly promising
in that they hold high relevance for practitioners, the breadth and complexity of the issue exceeded the scope of this thesis.
Future investigations into this topic could take a closer look at how this topic affects developers at different degrees
of experience in a language, as it seems like Implicitness might slow down experienced professionals, but make it harder
for newer developers to comprehend the code.

Questions on performance, aesthetics, and readability only made their way into the results in their relation to
idiomatic code, but each of these topics is deserving of significantly more attention in the context of the culture
surrounding a programming language community, or the tech community at large.
